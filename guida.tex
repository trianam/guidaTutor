\documentclass[11pt,a4paper]{article}
\usepackage[utf8x]{inputenc}
\usepackage[T1]{fontenc}
\usepackage[italian]{babel}
\usepackage{graphicx}
\usepackage[hyphens]{url}
\usepackage{hyperref}
%\usepackage{breakurl}
\usepackage{acronym}

\newacro{ISEE}{Indicatore  della Situazione Economica Equivalente}
\newacro{SOL}{Servizi On Line}

\begin{document}
\title{Guida per le matricole}
\author{Tutor di informatica}
\date{Settembre 2016}
\maketitle


In questa breve guida vengono introdotti alcuni servizi e procedure di uso 
comune che sono utili nel corso di studio. Per dettagli è bene
comunque riferirsi al sito dell'università \url{http://www.unifi.it/}
e in particolare al \emph{manifesto degli studi} che è reperibile
all'indirizzo
\url{http://www.unifi.it/vp-6385-manifesto-degli-studi.html}.

Il manifesto degli studi è il documento di riferimento per quanto
riguarda l'ambito normativo che regola lo svolgimento del corso di
laurea. In questo vi sono indicate tutte le procedure, scadenze e
normative che è bene conoscere per svolgere gli iter burocratici
necessari.

\section{Immatricolazione e \acs{ISEE}}
La prima cosa che uno studente deve fare prima di immatricolarsi è
recarsi in un centro CAF/INPS dove deve sottoscrivere la dichiarazione
sostitutiva unica per l'ottenimento dell'attestazione
dell'\ac{ISEE}. È importante fare questa operazione per tempo prima di
continuare con la procedura di immatricolazione.

È possibile immatricolarsi all'interno della specifica
finestra temporale indicata nel manifesto degli studi, la procedura si
svolge online accedendo al portale \ac{SOL}
(\url{http://sol.unifi.it}). Dopo aver fatto una 
registrazione preliminare, è necessario fornire i dati richiesti
e poi si potrà scaricare un bollettino MAV necessario per pagare la
prima tassa entro il termine indicato. Dopo circa 10 giorni dal
pagamento del bollettino, verranno assegnati allo studente matricola e
password da usare per accedere ai diversi servizi. Dopo di che è
necessario recarsi in segreteria e consegnare la domanda stampata
insieme ai documenti richiesti.

È possibile anche fare la procedura di immatricolazione con modalità
part time, in questo caso la durata del corso di laurea raddoppia (6
anni invece di 3) in quanto c'è un limite al numero di crediti (quindi
esami) che è
possibile dare in un anno. In cambio i contributi dovuti
all'università vengono ridotti.

Riferirsi comunque al manifesto degli studi per la procedura
dettagliata.

\section{Rimborsi per merito}
L'università prevede degli sconti, applicati alle tasse, per alcuni
corsi di laurea. Questi rimborsi vengono calcolati automaticamente e
sono comunicati allo studente sulla mail istituzionale. I parametri
con cui vengono calcolati questi rimborsi sono il numero di crediti
dati e la media degli esami. Per dettagli riferirsi al manifesto degli
studi.

\section{Tessera dello studente}
La tessera dello studente viene consegnata per posta a casa, e
contiene la matricola e un codice a barre di questa. È utile per
ottenere libri in prestito dalla biblioteca e per chiedere gli sconti
in alcuni musei, cinema, etc..

\section{Tessera della mensa}
Per accedere ai servizi di
ristorazione offerti dalla mensa è necessario possedere una tessera (è
possibile accedere anche da esterno, però il costo è maggiore). Il
costo dei pasti è stabilito secondo l'\ac{ISEE} presentato all'atto
di iscrizione. Per
ottenere tale tessera bisogna recarsi nell'apposito ufficio situato
nello stesso edificio della casa dello studente in Viale Morgagni 51,
al piano terra (la mensa è nel piano seminterrato).

\section{Portale \acf{SOL}}
Il portale \ac{SOL}, all'indirizzo \url{http://sol.unifi.it} è
un'importante risorsa, in quanto da qui è possibile accedere, tra le
altre cose, ai servizi
\begin{itemize}
\item immatricolazione,
\item visualizzazione dei dati dello studente e dello status delle iscrizioni,
\item riepilogo delle tasse pagate e scaricamento dei
  bollettini per pagarle,
\item prenotazione esami e verbalizzazione online,
\item scaricamento software gratuiti (Matlab e Microsoft).
\end{itemize}
In particolare è necessario ricordarsi di pagare le tasse entro le
scadenze, e di prenotare gli esami durante le specifiche finestre
temporali, prima di sostenerli.

Per accedere ai servizi è necessario inserire la matricola e
la password fornite (escluso per l'immatricolazione online).

\section{Piattaforma MOODLE}
La piattaforma MOODLE all'indirizzo \url{https://e-l.unifi.it} è il
principale strumento di comunicazione tra professori e
studenti. Questi viene usato per fornire tutte le informazioni
utili per il corso, come il programma, le modalità di esame, i testi
consigliati e le notizie. Viene anche usato per fornire
agli studenti il materiale del corso come le dispense e le slides. Le
pagine dei corsi vengono amministrate dai singoli professori,
quindi l'uso che viene fatto del portale può variare da corso a corso,
rimane comunque il principale punto di riferimento durante lo studio.

Per accedere a MOODLE è necessario inserire la matricola e la password
fornite. In seguito è necessario navigare fino a trovare i corsi di
interesse e registrarsi a questi inserendo la chiave fornita dal
professore a lezione (se per caso siete assenti la prima lezione,
chiedetela al professore o ai compagni). Le volte successive i corsi a
cui si è registrati si ritrovano dentro \emph{I miei corsi} nel
pannello \emph{Navigazione}.

\section{Portale OPAC}
Il portale OPAC all'indirizzo \url{http://opac.unifi.it} è lo
strumento con cui si cercano i libri che è possibile prendere in
prestito dalle varie biblioteche universitarie. Per accedere alla base
di dati non è necessario registrarsi, in home si presenta un form di
ricerca (semplice o avanzata) nel quale si compilano i campi del libro
desiderato. Cliccando su \emph{Vai} viene mostrata 
la lista dei libri trovati, e per ognuno di essi sulla destra viene
mostrato l'elenco delle biblioteche (con il numero di copie totali e
occupate) nelle quali è presente. Cliccando sulla biblioteca di
interesse viene mostrata la lista delle copie disponibili e occupate,
per ognuna di esse viene mostrata anche la collocazione da comunicare
al bibliotecario per effettuare il prestito. Il bibliotecario vi
richiederà la tessera dello studente o la matricola per convalidare il
prestito.

In alto è presente il link \emph{Scheda utente} che permette di
accedere alla pagina dei
prestiti in corso, oltre allo storico e alle eventuali
prenotazioni. Da qui è possibile estendere la durata dei prestiti in
corso, prima della data 
di scadenza di questi.

Per restituire i libri è necessario consegnarli semplicemente al
bibliotecario, ricordarsi che se questi non vengono consegnati in
tempo, c'è una penalità che consiste in giorni di blocco dei
prestiti.

\section{Mail istituzionale}
Ad ogni studente viene fornita una mail istituzionale costituita da
\nolinkurl{nome.cognome@stud.unifi.it} che può essere usata per le
comunicazioni con i professori, le segreterie o chiunque altro. La
mail rimane attiva per tutto il periodo di studi e fino a tre anni
dopo la laurea.

È possibile accedere all'interfaccia web all'indirizzo
\url{http://webmail.stud.unifi.it/} (o dal link sul portale \ac{SOL})
con la matricola e la password fornite. L'interfaccia web è
la stessa di Google.

È possibile anche configurare client di posta esterni seguendo le
istruzioni riportate all'indirizzo
\url{http://www.siaf.unifi.it/vp-987-istruzioni-per-l-accesso-tramite-client-di-posta-elettronica.html}.

\section{Proxy}
Il proxy dell'università serve per accedere da reti esterne, ad
esempio da casa, come se si fosse collegati all'interno della rete
universitaria. In particolare questo è utile per poter scaricare da
casa gli articoli dai portali delle riviste con cui l'università è
convenzionata.

Per configurare il browser in modo da usare il proxy è necessario
impostare come indirizzo \nolinkurl{proxy-auth.unifi.it} e come porta
\nolinkurl{8888}. In seguito, quando verranno richieste le credenziali, è
necessario inserire matricola e password fornite.

Per informazioni dettagliata su come configurare i diversi browser,
riferirsi alle istruzioni all'indirizzo
\url{http://www.siaf.unifi.it/vp-680-impostazioni-proxy-per-accedere-alla-rete-di-ateneo-dall-esterno.html}. Ricordarsi
di disattivare il proxy quando non è necessario.

\section{Connessione wireless}
L'università mette a disposizione la possibilità di collegarsi via
wireless con tutti i dispositivi alle reti sparse su tutto il
territorio di Firenze con ESSID:
\begin{itemize}
\item \nolinkurl{UnifiWiFi},
\item \nolinkurl{FirenzeWiFi},
\item \nolinkurl{AOUCUnifiWiFi},
\item \nolinkurl{eduroam}.
\end{itemize}

Per collegarsi è sufficiente impostare la rete su una delle precedenti
(senza chiavi WPA o WEP), e aprire il browser. Alla prima richiesta
HTTP a un sito si verrà reindirizzati su una pagina di autenticazione,
nella quale vanno inserite matricola e password fornite.


\end{document}

%%% Local Variables:
%%% mode: latex
%%% TeX-master: t
%%% End:
